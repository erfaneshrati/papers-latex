\section{Related work and comparison}

\textbf{General Task Offloading Frameworks.} There are existing prior arts focusing on offloading computation from the mobile to the cloud\cite{Oedessa, Comet, CloneCloud, MAUI, ExecutionJavaScript, Refactoring}. However, all these frameworks share a limiting feature that makes them impractical for computation partitioning of the DNN applications. 


These frameworks are programmer annotations dependent as they make decisions about pre-specified functions, whereas JointDNN makes scheduling decisions based on the model topology and mobile network specifications in run-time. Offloading in function level, cannot lead to efficient partition decisions due to layers of a given type within one architecture can have significantly different computation and data characteristics. For instance, a specific convolution layer structure can be computed on mobile or cloud in different models in the optimal solution. 

Neurosurgeon is the only prior art exploring a similar computation offloading idea in DNNs between the mobile device and the cloud server at layer granularity. Neurosurgeon assumes that there is only one data transfer point and the execution schedule of the efficient solution starts with mobile and then switches to the cloud, which performs the whole rest of the computations. Our results show this is not true especially for online training, where the optimal schedule of execution often follows the mobile-cloud-mobile pattern. Moreover, generative and autoencoder models follow a multi data transfer points pattern. Also, the execution schedule can start with the cloud especially in case of generative models where the input data size is large. Furthermore, inter-layer optimizations performed by DNN libraries are not considered in Neurosurgeon. Moreover, Neurosurgeon only schedules for optimal latency and energy, while JointDNN adapts to different scenarios including battery limitation, cloud server congestion, and QoS. Lastly, Neurosurgeon only targets simple CNN and ANN models, while JointDNN utilizes a graph based approach to handle more complex DNN architectures like ResNet and RNNs.